\documentclass[9pt]{article}
\usepackage{amsmath}
\usepackage{amssymb}
\usepackage{url}
\usepackage{multicol}
\usepackage{multirow}
\usepackage[usenames, dvipsnames]{xcolor}
\usepackage{nopageno}
\usepackage{hyperref}

\renewcommand{\arraystretch}{1.5}
\leftmargin=0.25in
\oddsidemargin=0.25in
\textwidth=6.0in
\topmargin=-1.2in
\textheight=10.00in

\raggedright
\pagestyle{empty}
%\pagenumbering{arabic}

\def\bull{\vrule height 0.8ex width .7ex depth -.1ex }

\newenvironment{changemargin}[2]{%
  \begin{list}{}{%
      \setlength{\topsep}{0pt}%
    \setlength{\leftmargin}{#1}%
    \setlength{\rightmargin}{#2}%
    \setlength{\listparindent}{\parindent}%
  \setlength{\itemindent}{\parindent}%
    \setlength{\parsep}{\parskip}%
    }%
  \item[]}{\end{list}
    }

\newcommand{\lineover}{
  \begin{changemargin}{-0.05in}{-0.10in}
    \vspace*{-9pt}
    \hrulefill \\
    \vspace*{-2pt}
  \end{changemargin}
}

\newcommand{\header}[1]{
  \begin{changemargin}{-0.5in}{-0.5in}
    \scshape{#1}\\
        \lineover
  \end{changemargin}
}

\newcommand{\name}[1]{
  %\begin{changemargin}{0.0in}{0.0in}
  %\begin{center}
  {\LARGE \scshape {#1}}
  %\end{center}
  %\lineover
  %\end{changemargin}
}

\newcommand{\contact}[4]{
  \begin{changemargin}{-0.65in}{-0.65in}
    \begin{multicols}{2}
      \name{{#1}}\vfill\null % this is name
                        \columnbreak
                        {#2}\\	
                        {#3}\\ 
                        {#4}\\ 
    \end{multicols}
  \end{changemargin}
}

\newenvironment{body} {
  \vspace*{-16pt}
        \begin{changemargin}{-0.6in}{-0.65in}
        }	
        {\end{changemargin}
}	


% END RESUME DEFINITIONS

\begin{document}
%%%%%%%%%%%%%%%%%%%%%%%%%%%%%%%%%%%%%%%%%%%%%%%%%%%%%%%%%%%%%%%%%%%%%%%%%%%%%%%%
% Name
\contact{Saurav Shekhar}
{{\hspace{172pt}}(+41) 76 410 31 18}
{{\hspace{130pt}} \href{mailto:sauravshekhar01@gmail.com}{sauravshekhar01@gmail.com}}
%{\hspace{190pt}Github: \href{http://github.com/sshekh}{\emph{{\textit{sshekh}}}}}
{{\hspace{180pt}} \href{https://sshekh.github.io/}{sshekh.github.io}}  

%%%%%%%%%%%%%%%%%%%%%%%%%%%%%%%%%%%%%%%%%%%%%%%%%%%%%%%%%%%%%%%%%%%%%%%%%%%%%%%%
% Education
\header{Education}

\vspace{4pt}
        \renewcommand{\arraystretch}{1}
        \begin{tabular}{rll}
          2016- & \textbf{MSc Computer Science},  ETH Z{\"u}rich  &- \\
                  & ESOP scholar, awarded to 3 admits among 150 in MSc Computer Science 2016 &\\\\
          2012-2016 & \textbf{B.Tech Computer Science},  Indian Institute of Technology, Kanpur & 9.1/10 \\ 
                  &  Academic Excellence Award for year 2012-13 & \\\\
        \end{tabular}

        %\header{Research Interests}
        %\header{Publications :(}
\header{Professional experience}
\begin{body}
  \vspace{14pt}
  
  \begin{description}
  	\item[\normalsize{Zero shot emoji recognizer}] \hfill 
  	\textit{June '18 - September '18} \\
  	\textit{Internship in Handwriting Recognition Research, Machine Perception, 
        Google Mountain View}
  \begin{itemize}
  	\item Researched Machine Learning models for learning better embeddings for emoji recognition
  	%\item (detail): Aim of the project is to design ML models that can classify handwriting
        %  strokes into different emoji classes using just emoji images during training. Experimented
        %  with image based models by rendering stroke data to image and training Similarity Networks
        %  for zero shot learning. Achieved 40\% top-25 accuracy on 1.6k emoji classes
  \end{itemize}
  	
  \end{description}

  \begin{description}
    \item[\normalsize{Speeding up model evaluation for black box optimization}] 
      \hfill \textit{March '18 - June '18} \\
      \textit{Internship at Amazon CoreAI, Berlin}
      \begin{itemize}
        \item Trained ML models on thousands of hyperparameter configurations and created
          lookup tables with extrapolation for epochwise simulation of black box
          model.
        %\item (detail) Trained Machine Learning models on thousands on hyperparameter configurations 
        %  generated with pseudo-grid sampling and collected the results for different 
        %  dataset types. Created lookup tables with extrapolation for epochwise
        %simulation of black box model on any given hyperparameter configuration.
        %Sped up evaluation of Hyperparameter Optimization algorithms by 
        %  several orders of magnitude using offline tables
      \end{itemize}

    \item[\normalsize{Recurrent Neural Networks on GPU for Particle Physics applications}] \hfill \textit{May '17 - August '17} \\
      \textit{Google Summer of Code project under CERN-HEP Software Foundation, Geneva}
      \begin{itemize}
        \item Restructured the Deep Learning module and added Recurrent Neural Network
          support in TMVA Module in the ROOT data analysis framework. Features include
          parsing network configurations, storing and loading weights, training data with GPU 
          (using CuBLAS), BLAS and multi-threaded training support for CPUs.
        %\item Restructured the entire Deep Learning module in TMVA to better 
        %  suit advanced deep learning architectures like CNNs, RNNs and 
        %  Autoencoders. Benchmarking on images dataset for event classification
        %  in particle physics.
      \end{itemize}

    %\item [\normalsize{Personalized feed algorithm}] \hfill \textit{May '16 - August '16} \\
    %  \textit{Internship at ShareChat, a vernacular language social media application, Bangalore} \\
    %  \begin{itemize}
    %    \item Analyzed user's activity on the app's home feed and designed a new
    %      wilson score based popularity metric for posts. Tried collaborative 
    %      filtering techniques for computing user-post relevance.
    %  \end{itemize}

    \item[\normalsize{Real-time Market Data Monitor}] \hfill \textit{May '15 - July '15} \\
      \textit{Summer internship at Goldman Sachs, Bangalore} 
      \begin{itemize}
        \item Implemented monitoring and alerting for latency spikes and various
          market data sanity checks and improved market data subscription
          %\item Technology Used: Java, JNI, Bash, Reuters RFA API
      \end{itemize}

  \end{description}

\end{body}

%\vspace{5 mm}
\header{Projects}
%\vspace{2 mm}
\begin{body}
  \vspace{14pt}

  \begin{description}
    \item[\normalsize{Dialogue generation using Seq2seq models}] \hfill \textit{April '17 - June '17} \\
      \textit{Course project in Natural Language Understanding under Prof. Thomas Hofmann}
      \begin{itemize}
        \item Starting with a basic sequence to sequence model for conversational
          agent, improved upon the baseline using attention models, stacked RNNs,
          mutual information (for diversifying output) and beam search.
        \item Introduced histograms of word frequency as a metric to evaluate 
          model performance (especially diversity) along with perplexity and BLEU 
          score.
      \end{itemize}

    \item[\normalsize{Twitter Sentiment Classification}] \hfill \textit{March '17 - July '17} \\
      \textit{Course project (Kaggle challenge) in Computational Intelligence Lab
      under Prof. Thomas Hofmann}
      \begin{itemize}
        \item Used GloVe and Word2Vec to extract word embeddings from tweets, 
          along with a combination of unsupervised sentence level embeddings to
          form feature vectors.
        \item Used an ensemble of LSTM, GRU and Convolutional Networks
          to achieve a classification accuracy of 89\% on a dataset of 2.5 million
          tweets. This was the {\bf top} scoring model on the leaderboard. 
      \end{itemize}

    \item[\normalsize{Optimized implementation of Latent Dirichlet Allocation}]  \hfill \textit{March '17 - June '17} \\
      \textit{Course project in How to Write Fast Numerical Code under Prof. Markus P{\"u}schel}
      \begin{itemize}
        \item Enhanced the performance of a standard Latent Dirichlet Allocation (LDA)
          implementation on single core using memory usage improvements, use of 
          SIMD instructions, cache analysis etc.
        \item Achieved a speedup of 11x over original C implementation written by authors
          of LDA.
      \end{itemize}

    %\item[\normalsize{Object detection and classification in Surveillance videos}] \hfill  \textit{Jan '16 - Apr '16} \\
    %  \textit{Course project in Machine Learning, tools and techniques under Prof. Harish Karnick, IIT Kanpur}
    %  \begin{itemize}
    %    \item Aim of the project was to detect and classify objects from the 
    %      university's survelliance videos into pre-defined classes (2/4 wheelers,
    %      pedestrian etc)
    %    \item Used selective search and background subtraction techniques to 
    %      extract candidate region proposals for classification. 
    %    \item Extracted features using SIFT, HOG, CNNs, Autoencoders and Restricted
    %      Boltzman Machines. Compared the classification accuracy obtained with 
    %      different classification algorithms.
    %  \end{itemize}

    %\item[\normalsize{Identifying age, gender and health from brain fMRI images}] \hfill  \textit{Sept '16 - Dec '16} \\
    % \textit{Course project (Kaggle challenges) in Machine Learning under Prof. Joachim Buhmann, ETH Zurich}
    %  \begin{itemize}
    %    \item Used segmentation (into white and grey matter) and extracted features like canny edges, SIFT, Histograms 
    %      from input brain fMRI images
    %    \item Used SVM's, Logistic Regression, Random Forests and ensembles like Adaboost, bagging to improve accuracy
    %  \end{itemize}

    %\item[\normalsize{Online algorithms for large datasets}] \hfill  \textit{Sept '16 - Dec '16} \\
    %  \textit{Course project in Data Mining under Prof. Andreas Krause, ETH Zurich}
    %\begin{itemize}
    %  \item Implemented online algorithms for similarity detection (Locality sensitive hashing), Clustering 
    %    (coreset construction followed by k-means), Image Classification (SVMs using Random Fourier Features) and 
    %    News recommendation (linUCB) 
    %  \item Placed in top 15\% on the leaderboards
    %\end{itemize}

    %\item[\normalsize{Robust PCA via Convex Optimization}] \hfill \textit{Apr '16} \\
    %  \textit{Term paper in Convex Optimization under Prof. Ketan Rajawat, IIT Kanpur}
    %  \begin{itemize}
    %  \item Studied and compared the current best algorithms for low rank matrix recovery like the 
    %    accelerated proximal
    %    gradient algorithm (APG), Augmented Lagrange Multiplier method (ALM), Dual method etc.
    %  \item Ran simulations with different size and error matrix and compared results of all algorithms 
    %    on different metrices 
    %      like time taken, reconstruction error, error-rate with iterations etc.
    %  \end{itemize}

    %\item[\normalsize{Random Graphs}]	\hfill 	\textit{July '15 - Nov '15} \\
    %  \textit{Undergraduate project under Prof. Surender Baswana, IIT Kanpur} 
    %  \hfill %\href{http://home.iitk.ac.in/~sshekh/cs498a/report.pdf}{\textit{\textcolor{ProcessBlue}{link to report}}}
    %\begin{itemize}
    %  \item Studied Erdos-Renyi phase transitions and expected linear time 
    %    algorithms for finding biconnected components in Random graphs.
    %  \item Worked on average case analysis of an incremental algorithm for 
    %    maintaining DFS tree in an undirected graph.
    %\end{itemize}

    %	  \item[\normalsize{Concurrent data Structures in Haskell}] \hfill  \textit{July '15 - Nov '15} \\
    %	  \textit{Course project in Functional Programming under Prof. Piyush Kurur, IIT Kanpur}
    %	  \begin{itemize}
    %	   % \item Aim was to develop a non-blocking queue data structure in Haskell 
    %	   \item Implemented Michael \& Scott's lock-free queue algorithm in Haskell.
    %	   \item Used atomic-primops package for CAS and other atomic operations.
    %	   \item Project developed as an open source Cabal Package.
    %	  \end{itemize}

    %	  \item[\normalsize{Intelligent Surveillance System}] \hfill  \textit{May '14 - July '14}\\
    %	  \textit{Summer project under Prof. Harish Karnick, IIT Kanpur}
    %	  \begin{itemize}
    %	  \item Involved improving the video surveillance system for traffic monitoring in the campus 
    %    of IIT Kanpur. Studied various methods for background subtraction and motion detection 
    %    (optical flow) for selecting candidate
    %		frames that contain useful data in the surveillance video
    %	  \item Implemented a real-time system for adaptive background subtraction using Gaussian Mixture Model
    %	  \item Extracted candidate license plate areas from the images and enhanced those using 
    %    morphological operations as a first
    %		step towards Optical Character Recognition
    %	  \item Implemented Viola-Jones object detection framework for vehicle classification using OpenCV. 		
    %	  \end{itemize}
\newpage
  \item[\normalsize{Scala to MIPS Assembly Compiler}] \hfill \textit{Jan '15 - Apr '15} \\
    \textit{Course Project in Compilers under prof. Subhajit Roy, IIT Kanpur}
  \begin{itemize}
    \item Programmed a Scala to MIPS cross compiler with support for basic 
      datatypes, conditional statements,
      looping statements, arrays, nested functions, recursion and object oriented 
      features.
    \item Awarded as the $2^{nd}$ best project for the course out of 22 teams.
  \end{itemize}

\item[\normalsize{Extension of NACHOS}]  \hfill \textit{Aug '14 - Nov '14} \\
  \textit{Course Project in Operating Systems under prof. Mainak Chaudhuri, IIT Kanpur}
  \begin{itemize}
    \item Extended the standard system call library of NachOS and implemented 
      Fork, Exec, Join, Yield, Sleep, Exit system calls. Implemented process 
      scheduling algorithms like UNIX scheduling, FIFO, Round robin, SJF and non-preemptive scheduling.
    \item Programmed page replacement algorithms: Random allocation, FIFO, LRU 
      and LRU-clock and evaluated relative performance.
  \end{itemize}

  \end{description}
  \smallskip
\end{body}
\vspace{2mm}
  %\header{Short Projects}
  %\vspace{2 mm}
  %\begin{body}
  %    \vspace{14pt}
  %	\begin{description}
  % 	  {\bf Yahoo HackU! 2013} 	\hfill \textit{(August '13)} \\
  % 	  \begin{itemize}
  % 	  \item
  % 	  \item Made a facebook app that could post a picture to any facebook post. Also made a chrome addon to add text to an image 
  % 	  and then upload that pic. 
  % 	  \item The user could make a meme of an image and then post it to facebook on the fly without having to make it on a different 
  % 	  site and then download it. 
  % 	  \item The posting part was done by using a Facebook Graph APIs and editing of image was done by HTML canvas.
  % 	  \end{itemize}
  % 	

  % 	  {\bf Oz programming language interpreter} \hfill \textit{(Sept '14)} \\
  % 	  \textit{Course project in Principles of Programming Languages(CS350) under Prof. Satyadev Nandakumar, IIT Kanpur}
  % 	  \begin{itemize}
  % 	  \item
  % 	  \item Implemented a meta-circular interpreter for a declarative sequential model of Oz.
  % 	  \item Implemented the semantic stack and single assignment store using an easy-to-parse abstract syntax tree.
  % %	  \item Also studied the declarative concurrent model of Oz programming language.
  % 	  \end{itemize}
  % 	  
  %	  %\href{http://hackyourworld.org/~saurav/takneek/}
  %  {\bf Takneek 2013, IIT Kanpur}	\hfill \textit{(Aug '13)} \\  
  %  \begin{itemize}
  %  \item Implemented an applicaton to search through facebook timeline of a given friend of the user or the user himself based on the given 
  %  time interval and keywords and auto comment on the search results.
  %  \item Used Facebook's SDK and fuse.js for robust search. Won 1st place.
  %  \end{itemize}

  %	  {\bf Cuckoo Hashing} \hfill \textit{Apr '15} \\
  %	  \textit{Term paper in Randomized Algorithms under Prof. Surender Baswana, IIT Kanpur}
  %	  \begin{itemize}
  %	  \item 
  %	   \item Studied Cuckoo Hashing and its average case runtime analysis
  %	  \end{itemize}

  %	\end{description}
  %	\smallskip
  %\end{body}

\vspace{2 mm}
\header{Academic}
\begin{body}
  \vspace{14pt}
        \begin{changemargin}{0.15in}{0.15in}
          \begin{itemize}
            \item Research Assistant, Information Science and Engineering Group, ETH Z{\"u}rich, fall 2017
            \item Teaching Assistant, Data Structures and algorithms, IIT Kanpur, fall 2015
            \item Teaching Assistant, Advanced algorithms, IIT Kanpur, fall 2015
            \item Secured All India Rank 958 in IIT-JEE among 500,000 candidates.
            \item Selected in Top 1\% of each of the National Standard Examinations 
              in Physics (NSEP), Chemistry (NSEC) and Astronomy (NSEA) and Regional
              Mathematics Olympiad.
          \end{itemize}
        \end{changemargin}
\end{body}


%%%%%%%%%%%%%%%%%%%%%%%%%%%%%%%%%%%%%%%%%%%%%%%%%%%%%%%%%%%%%%%%%%%%%%%%%%%%%%%%
% Relevant Courses
% \vspace{3 mm}
% \header{Relevant Courses}
% 
% \vspace{3 mm}
% \begin{body}
% 	\vspace{4pt}
% 	\begin{changemargin}{0.15in}{0.15in}
% 	\begin{multicols}{3}
%         Operating Systems  \\
%         Principles of Programming languages \\
% 	Functional Programming \\
% 	Applied Stochastic Processes \\
% 	Probabilistic Machine Learning \\
% 	Convex Optimization \\
% 	Algorithms Lab $^{*}$ \\
% 	Machine Learning $^{*}$ \\
% 	Data Mining $^{*}$ \\
% 	Probabilistic AI $^{*}$ \\
% 	Advanced Systems lab $^{*}$
% 	\end{multicols}
% 	\end{changemargin}
% 	\footnotesize $^{*}$ Expected to be completed by Feb' 17
% \end{body}

\vspace{2 mm}
\header{Miscellanous}
%\vspace{2 mm}
\begin{body}
  \vspace{14pt}
        \begin{changemargin}{0.15in}{0.15in}
          \begin{itemize}
            \item Google Hash Code 2017 onsite Finalist, under top $50$ worldwide 
              among 3000 teams in the qualification round.
            \item Ranked $10^{th}$ among 60 teams and awarded a bronze medal at 
              ACM ICPC SWERC 2016 Regionals.
            \item $6^{th}$ place among top 25 teams selected (among 8000 in online 
              rounds) for the Codechef Snackdown onsite finale 2015. \\
            \item Secretary, Programming Club (\textit{2013-2014}) - Assisted in
              organizing Programming competitions 
              and took introductory programming lecture for the freshers.
              %\item Worked with a team of 5 members on creating and testing the 
              %problems for IOPC, 
              %the annual 24 hr long algorithmic programming contest during 
              %	Techkriti 2015. 901 teams participated in the contest.
            \item Advanced to Round 2 of Facebook Hackercup 2013 (position 287), 
              and Round 2 of Google Code Jam 2014. \\
              %\item Secured position 11 and 20 among 6000 Indian participants in Codechef July 
              %challenge and June challenge 2014 respectively \\
            \item Secured $1^{st}$ place in IHPC, high performance computing contest 
              and $2^{nd}$ place in Battlecity, AI bot programming challenge in 
              Techkriti 14. \\
              %	  time interval and keywords and auto comment on the search results.
              %	  \item Used Facebook's SDK and fuse.js for robust search. Won 1st place.
              %\item Attended the large scale optimization for machine learning summer school, 2016 at IISC, Bangalore \\
              %\item Attended the mini symposium on Computation and Optimization held at IIT Kanpur \\
              %\item Linkedin Profile \href{https://www.linkedin.com/in/saurav-shekhar-16786958} {saurav-shekhar-16786958} \\
          \end{itemize}
        \end{changemargin}
\end{body}
%%%%%%%%%%%%%%%%%%%%%%%%%%%%%%%%%%%%%%%%%%%%%%%%%%%%%%%%%%%%%%%%%%%%%%%%%%%%%%%%
% Skills
\vspace{3 mm}
\header{Technical Skills}

\vspace{3 mm}
\begin{body}
  \vspace{14pt}
        \begin{changemargin}{0.15in}{0.15in}
          \emph{\textbf{Coursework:}} Computer Vision, Probabilistic Machine Learning,
          Graphical models for image analysis (\emph{Ongoing})
          Deep Learning, Natural Language Understanding, Machine 
          Learning, Hardware Architectures in Machine Learning, 
          How to Write Fast Numerical Code, Algorithms Lab \\
          \emph{\textbf{Programming Languages: }}{} C++ (Proficient), Python (Proficient), 
          Lua, Java, Haskell, Bash Shell\\
          \emph{\textbf{Software: }}{} Torch, Tensorflow, CUDA (cuBLAS), MATLAB, 
          Caffe, MySQL, GIT, MIPS Assembly, \LaTeX \\
        \end{changemargin}
\end{body}

\end{document}
