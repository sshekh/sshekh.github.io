\documentclass[9pt]{article}
\usepackage{amsmath}
\usepackage{amssymb}
\usepackage{url}
\usepackage{multicol}
\usepackage[usenames, dvipsnames]{xcolor}
\usepackage{nopageno}
\usepackage{hyperref}

\renewcommand{\arraystretch}{1.5}
\leftmargin=0.25in
\oddsidemargin=0.25in
\textwidth=6.0in
\topmargin=-1.2in
\textheight=10.00in

\raggedright
\pagestyle{empty}
%\pagenumbering{arabic}

\def\bull{\vrule height 0.8ex width .7ex depth -.1ex }

\newenvironment{changemargin}[2]{%
  \begin{list}{}{%
    \setlength{\topsep}{0pt}%
    \setlength{\leftmargin}{#1}%
    \setlength{\rightmargin}{#2}%
    \setlength{\listparindent}{\parindent}%
    \setlength{\itemindent}{\parindent}%
    \setlength{\parsep}{\parskip}%
  }%
  \item[]}{\end{list}
}

\newcommand{\lineover}{
	\begin{changemargin}{-0.05in}{-0.10in}
		\vspace*{-9pt}
		\hrulefill \\
		\vspace*{-2pt}
	\end{changemargin}
}

\newcommand{\header}[1]{
	\begin{changemargin}{-0.5in}{-0.5in}
		\scshape{#1}\\
  	\lineover
	\end{changemargin}
}

\newcommand{\name}[1]{
	%\begin{changemargin}{0.0in}{0.0in}
	\begin{center}
	  	{\LARGE \scshape {#1}}
	\end{center}
                %\lineover
	%\end{changemargin}
}

\newcommand{\contact}[5]{
	\begin{changemargin}{-0.65in}{-0.65in}
		\begin{multicols}{2}
			{#1}\\ \smallskip 
			{#2}\\ \smallskip
			{#3}\\ \smallskip
			{#4}\\ \smallskip 
			{#5}\\ \smallskip
		\end{multicols}
	\end{changemargin}
}

\newenvironment{body} {
	\vspace*{-16pt}
	\begin{changemargin}{-0.6in}{-0.65in}
  }	
	{\end{changemargin}
}	

\newcommand{\school}[4]{
	\textbf{#1} \hfill \emph{#2\\}
	#3\\ 
	#4\\
}

% END RESUME DEFINITIONS

\begin{document}

%%%%%%%%%%%%%%%%%%%%%%%%%%%%%%%%%%%%%%%%%%%%%%%%%%%%%%%%%%%%%%%%%%%%%%%%%%%%%%%%
% Name
\name{Saurav Shekhar}
\contact{MSc Computer Science}{Department of Informatik}{ETH Zurich}
{{\hspace{130pt}}sauravshekhar01@gmail.com} {{\hspace{180pt}}(+41) 764103118}
 
%%%%%%%%%%%%%%%%%%%%%%%%%%%%%%%%%%%%%%%%%%%%%%%%%%%%%%%%%%%%%%%%%%%%%%%%%%%%%%%%
% Education
\header{Education}

\vspace{4pt}
	\begin{tabular}{ | l | l | l | l | }
	\hline	
	\textbf{Year} & \textbf{Degree/Certificate} & \textbf{Institute} & \textbf{CGPA/Percentage} \\ \hline
	2018 (expected) & Masters of Science & ETH Z{\"u}rich & - \\ \hline
	2016 & Bachelor of Technology & Indian Institute of Technology, Kanpur & 9.1/10 \\ \hline
	%2012 & AISSCE, XII (CBSE) & Jawahar Vidya Mandir, Ranchi (C.B.S.E.) & 95.2\%  \\ \hline
	%2010 & AISSE, X (CBSE) & DAV Public School, Gumla(C.B.S.E.) & 10.0/10.0  \\ \hline
	\end{tabular}

% Academic Achievements
%\vspace{3 mm}

\header{Internships}
\begin{body}
 \vspace{14pt}
  \begin{description}
  \item[\normalsize{Real-time Market Data Monitor}] \hfill \textit{May '15 - July '15} \\
  \textit{Summer internship at Goldman Sachs, Bangalore} \\
  Application is capable of consuming market data from various sources. 
  Functionality includes monitoring and alerting on stale, missing or malformed data
  \begin{itemize}
   \item Implemented monitoring and alerting for latency spikes and various market data sanity checks
   \item Improved market data subscription
   \item Abstracted out various monitoring and alerting functionalities so that these can be reused across multiple market data source systems
   \item Technology Used: Java, JNI, Bash, Reuters RFA API
  \end{itemize}
  \end{description}

 \begin{description}
   \item [\normalsize{Personalized feed algorithm}] \hfill \textit{May '16 - August '16} \\
   \textit{Internship at ShareChat, a vernacular language social media startup, Bangalore} \\
    Aim of the project was to analyze user's activity on the app's home feed and design 
    a personalized feed algorithm for 700k active users of the application.
    \begin{itemize}
     \item Designed a new wilson score based popularity metric for posts.  \\ 
     \item Experimented with collaborative filtering techniques like implicit-feedback matrix factorization, user-user and item-item similarity
	   for computing user-post rating and developed methods to compute user-tag relevance.
     \item Used RankSVM to compute a linear model for user-post relevance.
    \end{itemize}
  \end{description}
  
\end{body}

%\vspace{5 mm}
\header{Projects}
%\vspace{2 mm}
\begin{body}
	\vspace{14pt}
	
	\begin{description}
      
     \item[\normalsize{}] 
      
	 \item[\normalsize{Object detection and classification in Surveillance videos}] \hfill  \textit{Jan '16 - Apr '16} \\
	  \textit{Course project in Machine Learning, tools and techniques under Prof. Harish Karnick, IIT Kanpur}
	  \begin{itemize}
            \item Aim of the project was to detect and classify objects from the university's survelliance videos into 
                  pre-defined classes (2/4 wheelers, pedestrian etc)
            \item Used selective search and background subtraction techniques to extract candidate region proposals for classification
            \item Extracted feature using various methods like SIFT, HOG, Convolutional Neural Networks, Autoencoders and Restricted
                  Boltzman Machines. Used classification algorithms like Random Forests, SVM and Decision trees and compared the results.
            \item On image classification, achieved an accuracy of more than 90\% using CNNs.
	  \end{itemize}

	 \item[\normalsize{Identifying age, gender and health from brain fMRI images}] \hfill  \textit{Sept '16 - Dec '16} \\
           \textit{Course project (Kaggle challenges) in Machine Learning under Prof. Joachim Buhmann, ETH Zurich}
	  \begin{itemize}
            \item Used segmentation (into white and grey matter) and extracted features like canny edges, SIFT, Histograms 
              from input brain fMRI images
            \item Used SVM's, Logistic Regression, Random Forests and ensembles like Adaboost, bagging to improve accuracy
	  \end{itemize}

	 \item[\normalsize{Online algorithms for large datasets}] \hfill  \textit{Sept '16 - Dec '16} \\
           \textit{Course project in Data Mining under Prof. Andreas Krause, ETH Zurich}
	  \begin{itemize}
            \item Implemented online algorithms for similarity detection (Locality sensitive hashing), Clustering 
              (coreset construction followed by k-means), Image Classification (SVMs using Random Fourier Features) and 
              News recommendation (linUCB) 
            \item Placed in top 15\% on the leaderboards
	  \end{itemize}

        %\item[\normalsize{Robust PCA via Convex Optimization}] \hfill \textit{Apr '16} \\
	%  \textit{Term paper in Convex Optimization under Prof. Ketan Rajawat, IIT Kanpur}
	%  \begin{itemize}
	%  \item Studied and compared the current best algorithms for low rank matrix recovery like the accelerated proximal
        %    gradient algorithm (APG), Augmented Lagrange Multiplier method (ALM), Dual method etc.
        %  \item Ran simulations with different size and error matrix and compared results of all algorithms on different metrices 
        %      like time taken, reconstruction error, error-rate with iterations etc.
	%  \end{itemize}

            \pagebreak
	  \item[\normalsize{Random Graphs}]	\hfill 	\textit{July '15 - Nov '15} \\
            \textit{Undergraduate project under Prof. Surender Baswana, IIT Kanpur} 
            \hfill \href{http://home.iitk.ac.in/~sshekh/cs498a/report.pdf}{\textit{\textcolor{ProcessBlue}{link to report}}}
	  \begin{itemize}
	   \item Studied Erdos-Renyi phase transitions and expected linear time algorithms for finding 
                biconnected components in Random graphs.
	   \item Worked on average case analysis of an incremental algorithm for maintaining DFS tree in an undirected graph 
	  \end{itemize}
	  
	  \item[\normalsize{Concurrent data Structures in Haskell}] \hfill  \textit{July '15 - Nov '15} \\
	  \textit{Course project in Functional Programming under Prof. Piyush Kurur, IIT Kanpur}
	  \begin{itemize}
	   % \item Aim was to develop a non-blocking queue data structure in Haskell 
	   \item Implemented Michael \& Scott's lock-free queue algorithm in Haskell
	   \item Used atomic-primops package for CAS and other atomic operations
	   \item Project developed as an open source Cabal Package
	  \end{itemize}
	
%	  \item[\normalsize{Intelligent Surveillance System}] \hfill  \textit{May '14 - July '14}\\
%	  \textit{Summer project under Prof. Harish Karnick, IIT Kanpur}
%	  \begin{itemize}
%	  \item Involved improving the video surveillance system for traffic monitoring in the campus of IIT Kanpur. Studied various methods for background subtraction and motion detection (optical flow) for selecting candidate
%		frames that contain useful data in the surveillance video
%	  \item Implemented a real-time system for adaptive background subtraction using Gaussian Mixture Model
%	  \item Extracted candidate license plate areas from the images and enhanced those using morphological operations as a first
%		step towards Optical Character Recognition
%	  \item Implemented Viola-Jones object detection framework for vehicle classification using OpenCV. 		
%	  \end{itemize}

% 	\item[\normalsize{Scala to MIPS Assembly Compiler}] \hfill \textit{Jan '15 - Apr '15} \\
% 	\textit{Course Project in Compilers under prof. Subhajit Roy, IIT Kanpur}
% 	\begin{itemize}
% 	 \item Programmed a Scala to MIPS cross compiler with support for basic datatypes, conditional statements,
% looping statements, arrays, nested functions, recursion and object oriented features
% 	 \item Awarded as the second best project for the course out of 22 teams
% 	\end{itemize}
% 	
	\item[\normalsize{Extension of NACHOS}]  \hfill \textit{Aug '14 - Nov '14} \\
	\textit{Course Project in Operating Systems under prof. Mainak Chaudhuri, IIT Kanpur}
	\begin{itemize}
	 \item Extended the standard system call library of NachOS and implemented Fork, Exec, Join, Yield, Sleep, Exit system calls
	 \item Implemented process scheduling algorithms like UNIX scheduling, FIFO, Round robin, SJF and non-preemptive scheduling and assessed the results
	 \item Programmed page replacement algorithms: Random allocation, FIFO, LRU and LRU-clock and evaluated relative performance
	\end{itemize}
	
	 \item[\normalsize{Advances in Generative models with deep learning}] \hfill  \textit{Jan '16 - Apr '16} \\
	  \textit{Course project in Probabilistic Machine Learning under Prof. Piyush Rai, IIT Kanpur}
	  \begin{itemize}
            \item Studied two recent generative models that use deep learning techniques, deep exponential families
                  and generative adversarial networks.  
            \item Used an implemntation of Deep Exponential Families for topic modelling on KOS blog entries from the
                  UCI ML Bag of Words dataset
            \item Used a Theano implementation of GAN to generate images from MNIST and CIFAR datasets.
	  \end{itemize}

	 
	 \item[\normalsize{R on Hadoop}]	\hfill 	\textit{Aug '14 - Nov '14} \\
	 \textit{Course project in CS52 under prof. Arnab Bhattacharya, IIT Kanpur}
	 \begin{itemize}
	  \item Setup a Hadoop cluster on an IBM Bladeserver and install RHadoop packages on the server 
	  \item Allowed distributed processing of R-code on the Hadoop cluster
	  \item This project was selected as one of the best in the course
	 \end{itemize}
 
% 	   \item[\normalsize{Projection algorithms for feasibility of Linear Programs}] \hfill  \textit{(May '14 - July '14)}\\
% 	   \textit{Summer project under prof. S K Mehta}
% 	   \begin{itemize}
% 	   \item Studied various Linear programming algorithms like simplex and interior point methods.
% 	   \item Tried heuristic methods for finding the vector of fastest convergence.
% 	   \item Studied Relaxation method for convex feasibility by Agmon, Motzkin and Schoenberg.
% 	  \end{itemize}
%	  \item[\normalsize{Advanced audio equalizer}]  \hfill \textit{May '13 - June '13}\\
%	  \textit{Summer Project under Programming Club, IIT Kanpur} 
%	  \begin{itemize}
%	  \item Devloped an audio equalizer in Java that used Fourier transform to break a song into its frequency range. 
%	  \item Variable ranges of frequency could be filtered out accurately using GUI.
%	  \item Used Minim Audio library and Processing development Enviroment. 
%	  \end{itemize}
%\clearpage
	 \end{description}
	\smallskip
\end{body}
	  
%\header{Short Projects}
%\vspace{2 mm}
%\begin{body}
%	\vspace{14pt}
%	\begin{description}
% 	  {\bf Yahoo HackU! 2013} 	\hfill \textit{(August '13)} \\
% 	  \begin{itemize}
% 	  \item
% 	  \item Made a facebook app that could post a picture to any facebook post. Also made a chrome addon to add text to an image 
% 	  and then upload that pic. 
% 	  \item The user could make a meme of an image and then post it to facebook on the fly without having to make it on a different 
% 	  site and then download it. 
% 	  \item The posting part was done by using a Facebook Graph APIs and editing of image was done by HTML canvas.
% 	  \end{itemize}
% 	
	  
% 	  {\bf Oz programming language interpreter} \hfill \textit{(Sept '14)} \\
% 	  \textit{Course project in Principles of Programming Languages(CS350) under Prof. Satyadev Nandakumar, IIT Kanpur}
% 	  \begin{itemize}
% 	  \item
% 	  \item Implemented a meta-circular interpreter for a declarative sequential model of Oz.
% 	  \item Implemented the semantic stack and single assignment store using an easy-to-parse abstract syntax tree.
% %	  \item Also studied the declarative concurrent model of Oz programming language.
% 	  \end{itemize}
% 	  
%	  %\href{http://hackyourworld.org/~saurav/takneek/}
%	  {\bf Takneek,IITK 2013}	\hfill \textit{(Aug '13)} \\  
%	  \begin{itemize}
%	  \item
%	  \item Implemented an applicaton to search through timeline of a given friend of the user or the user himself based on the given 
%	  time interval and keywords and auto comment on the search results.
%	  \item Used Facebook's SDK and fuse.js for robust search. Won 1st place.
%	  \end{itemize}
	  
%	  {\bf Cuckoo Hashing} \hfill \textit{Apr '15} \\
%	  \textit{Term paper in Randomized Algorithms under Prof. Surender Baswana, IIT Kanpur}
%	  \begin{itemize}
%	  \item 
%	   \item Studied Cuckoo Hashing and its average case runtime analysis
%	  \end{itemize}
	  
%	  {\bf Planar Graph Visualisation} \hfill \textit{Jan '13 - Apr '13} \\ 
%	  \textit{Semester project under ACA, IIT Kanpur}  
%	  \begin{itemize}
%	  \item
%	  \item Studied various methods like Coffman-Graham algorithm to minimize crossings in Planar Graph Drawing
%	  \item Studied basic Graph Theory along with various properties of Planar Graphs and heuristics of Graph drawing
%	  \end{itemize}
	  
% 	  {\bf Roulette} \hfill {\it (August '12)} \\ 
% 	  Electrovate 12, circuit design competition under Electronics club	
% 	  \begin{itemize}
% 	  \item
% 	   \item Designed and implemented a mini roulette using a breadboard and basic logic gates.
% 	   \item Used {\bf XOR gates} to implement flip-flop. Won {\bf most innovative circuit} award.
% 	  \end{itemize}

	  
%	\end{description}
%	\smallskip
%\end{body}

\header{Academic}
%\vspace{2 mm}
\begin{body}
	\vspace{14pt}
	\begin{changemargin}{0.15in}{0.15in}
	\begin{itemize}
	  \item ETH Excellence Scholarship and Opportunity award, awarded to 3 students among $\approx$ 150 students
            admitted to MSc program in the year 2016
	  \item Academic Excellence Award, IIT Kanpur, 2012-13 
	  \item Teaching Assistant for Data Structures and algorithms course, year 2015-16.
	  %\item Secured All India Rank {\bf 958} in {\bf IIT-JEE} among 500,000 candidates.
	  %\item Secured All India Rank {\bf 554} in {\bf AIEEE} among 1,000,000 candidates.
	  %\item Qualified Regional Mathematical Olympiad(RMO) 2012
	  \item Selected in Top 1\% of each of the National Standard Examinations in Physics (NSEP), Chemistry (NSEC) and Astronomy (NSEA) and Regional Mathematics Olympiad.
	\end{itemize}
	\end{changemargin}
\end{body}


%%%%%%%%%%%%%%%%%%%%%%%%%%%%%%%%%%%%%%%%%%%%%%%%%%%%%%%%%%%%%%%%%%%%%%%%%%%%%%%%
% Relevant Courses
% \vspace{3 mm}
% \header{Relevant Courses}
% 
% \vspace{3 mm}
% \begin{body}
% 	\vspace{4pt}
% 	\begin{changemargin}{0.15in}{0.15in}
% 	\begin{multicols}{3}
%         Operating Systems  \\
%         Principles of Programming languages \\
% 	Functional Programming \\
% 	Applied Stochastic Processes \\
% 	Probabilistic Machine Learning \\
% 	Convex Optimization \\
% 	Algorithms Lab $^{*}$ \\
% 	Machine Learning $^{*}$ \\
% 	Data Mining $^{*}$ \\
% 	Probabilistic AI $^{*}$ \\
% 	Advanced Systems lab $^{*}$
% 	\end{multicols}
% 	\end{changemargin}
% 	\footnotesize $^{*}$ Expected to be completed by Feb' 17
% \end{body}

\vspace{1 mm}
\header{Miscellanous}
%\vspace{2 mm}
\begin{body}
	\vspace{14pt}
	\begin{changemargin}{0.15in}{0.15in}
	\begin{itemize}
          \item $1^{st}$ place from the ETH Hub (top $50$ worldwide among 3000 teams) in Google Hash Code online Qualification round.
              Selected for the finals to be held at Google Paris in April 2017.
          \item Ranked $10^{th}$ among 60 teams and awarded a bronze medal at ACM ICPC SWERC 2016 Regionals
	  \item $6^{th}$ place among top 25 teams selected (among 8000 in online rounds) for the Codechef Snackdown onsite finale 2015 \\
	  %\item Secretary, Programming Club ( \emph{Academic Year 2013-2014} ) - Assisted in organizing Programming competitions 
	   % and took introductory programming lecture for the freshers.
	  %\item Worked with a team of 5 members on creating and testing the problems for IOPC, the annual 24 hr long algorithmic programming contest during 
	%	Techkriti 2015. 901 teams participated in the contest.
 	  \item Advanced to Round 2 of Facebook Hackercup 2013 (position 287), and Round 2 of Google Code Jam 2014 \\
	  %\item Secured position 11 and 20 among 6000 Indian participants in Codechef July challenge and June challenge 2014 respectively \\
	  \item Secured $1^{st}$ place in IHPC, high performance computing contest and $2^{nd}$ place in Battlecity, AI bot programming challenge in Techkriti 14. \\
%	  time interval and keywords and auto comment on the search results.
%	  \item Used Facebook's SDK and fuse.js for robust search. Won 1st place.
	 %\item Attended the large scale optimization for machine learning summer school, 2016 at IISC, Bangalore \\
         %\item Attended the mini symposium on Computation and Optimization held at IIT Kanpur \\
	    %\item Github Handle \href{http://github.com/sshekh} {\emph{{\textit{\textcolor{ProcessBlue}{sshekh}}}}} \\
	    %\item Linkedin Profile \href{https://www.linkedin.com/in/saurav-shekhar-16786958} {saurav-shekhar-16786958} \\
	\end{itemize}
        \end{changemargin}
\end{body}
%%%%%%%%%%%%%%%%%%%%%%%%%%%%%%%%%%%%%%%%%%%%%%%%%%%%%%%%%%%%%%%%%%%%%%%%%%%%%%%%
% Skills
\vspace{3 mm}
\header{Technical Skills}

\vspace{3 mm}
\begin{body}
	\vspace{14pt}
	\begin{changemargin}{0.15in}{0.15in}
	\emph{\textbf{Programming Languages: }}{} C, C++ (Proficient), Java, Python, Haskell, Perl, HTML, PHP, Bash Shell Scripting, nodejs\\
	\emph{\textbf{Software: }}{}  Torch7, Tensorflow, Caffe, MySQL, MongoDB, GIT, MIPS Assembly, \LaTeX, Gnuplot, Octave, MATLAB, Amazon Web Service \\
	\end{changemargin}
\end{body}

\end{document}
